\documentclass{article}

\usepackage[utf8]{inputenc}
\usepackage{lipsum}
\usepackage{listings}
\usepackage{mdwlist}
\usepackage{fancyhdr}

\pagestyle{fancy}
\fancyhf{}
\fancyhead[R]{Technical University of Denmark}
\fancyhead[L]{02239 -- Data security}
\fancyfoot[R]{\thepage}
\fancyfoot[L]{s070145 -- Jens Henrik Skuldbøl}

\title{Access Control Lab}
\author{Jens Henrik Skuldbøl, s070145}
\date{December 4th, 2013}

\begin{document}

\maketitle
\thispagestyle{empty}

\begin{quote}
	The implemented solutions in this assignment were discussed
	with Brian Lynnerup Pedersen, s042457, during development.
\end{quote}

\newpage

\section{Practical information}

\subsection{Technical notes}
A very small part of this program uses Java's New I/O
(\texttt{java.nio}) library, and therefor required Java 7
to run. It is, however, only the actual reading of files
that uses this library, so one should be able to run
the program using Java 6 with few modifications.

The code that uses \texttt{java.nio} is located in
\texttt{sample.Sampleaction.java}, lines 111-125.

\subsection{Building and running the program}
Included in the project are two shell-files, \texttt{build.sh}
and \texttt{run.sh}. Running either of these two files
does what their name implies.

These files were created and tested on a system running
Debian Linux, and I can make no guarantees that they
will work on other systems.

\subsection{Users}
The users implemented in this system are the ones listed
in the assignment. These users are
\begin{itemize}
	\item Alice (Basic subscription)
	\item Bob (Basic subscription)
	\item Carol (Silver subscription)
	\item David (Gold subscription)
	\item Elisa (Admin)
\end{itemize}

In keeping with realisticly insecure password standards,
each user's password is her name in lowercase letters.

\section{Design}

\subsection{Filesystem structure}
The movies in the system is arranged into folders
named by the subscription name (\texttt{basic},
\texttt{silver}, and \texttt{gold})
in a folder named \texttt{movies}. Each of these
folders contains the files (movies) available
to that subscription.

\subsection{System functionality}
As the purpose of this excercise is to demonstrate
Java's access control functionality, it's functionality
is rather limited.
Once the user logs in, the system adds a
\texttt{StreamingPrincipal} containing
the name of the users subscription to her
subject.

The \texttt{SampleAction} run then attempts
to read each of the directories containing movies,
adding them to a list of movies available to
the user if her subscription allows her to 
read the files.

After gathering available files, the system 
checks whether the user has writing priviledges
(making her admin).

Lastly, the program lists the movies available
(files readable) to the user, and then reads
them (printing their content).

\section{Implementation}

\subsection{Priviledges}
The policies specified in the assignment ranks four
subscriptions/roles in the following order:
\begin{enumerate}
	\item Basic
	\item Silver
	\item Gold
	\item Admin
\end{enumerate}
where each role includes the permissions of the previous.
The admin-role stands out by being the only one
who has writing permissions.

Due to the simple nature of this structure,
each role has all the needed permissions
specified in the policy-file.

Another solution would have been to specify
the unique permissions to a principal for each
role. Subjects with greater subscriptions/roles
would then have a principal for their role, and
for each one beneath it in the hierarchy.
This would make for a more flexible solution
allowing for more complex systems.
This was, however, not needed for a system of
this level of complexity.

\subsection{User authentication}
For some reason, the authentication implemented
in the previous lab did not want to play nice
with the code in this project.
As the focus of this lab is authorization
and not authentication, I've implemented a simple
method for authentication users, and check
their subscription.

This solution has usernames and passwords
hardcoded. When a user enters a correct username
and password, the method returns the
\texttt{RoleEnum} corresponding to that
users subscription.

This role is then saved in a private member in
the \texttt{SampleLoginModule}, which contains
the method as well, and is used to create the
principal in the \texttt{commit} method on
a successful login.

Had the previous implementation worked,
the users subscription should have been
stored in the password file along with
usernames and passwords.
This would have been done i a way similar
to how groups are handled in UNIX systems.

%\section{Task 5}

\newpage
\appendix

\section{Code}

\subsection{Task 3}

\subsubsection{Roles}
\lstinputlisting
[
	basicstyle=\footnotesize,
	numbers=left,
	numberstyle=\tiny,
	breaklines=true,
	language=Java
]
{../sample/RoleEnum.java}

\subsubsection{Principal}
\lstinputlisting
[
	basicstyle=\footnotesize,
	numbers=left,
	numberstyle=\tiny,
	breaklines=true,
	language=Java
]
{../sample/principal/StreamingPrincipal.java}

\subsubsection{Adding principal}
\lstinputlisting
[
	basicstyle=\footnotesize,
	numbers=left,
	numberstyle=\tiny,
	breaklines=true,
	language=Java,
	firstline=265,
	lastline=290
]
{../sample/module/SampleLoginModule.java}

\subsubsection{Action}
\lstinputlisting
[
	basicstyle=\footnotesize,
	numbers=left,
	numberstyle=\tiny,
	breaklines=true,
	language=Java
]
{../sample/SampleAction.java}

\subsection{Task 4}

\subsubsection{Policies}
\lstinputlisting
[
	basicstyle=\footnotesize,
	numbers=left,
	numberstyle=\tiny,
	breaklines=true,
	language=Java
]
{../sampleazn.policy}

\subsubsection{User authentication}
\lstinputlisting
[
	basicstyle=\footnotesize,
	numbers=left,
	numberstyle=\tiny,
	breaklines=true,
	language=Java,
	firstline=122,
	lastline=152
]
{../sample/module/SampleLoginModule.java}

\end{document}
